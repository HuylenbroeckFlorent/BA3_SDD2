\documentclass[10pt]{article}
\usepackage{fullpage}
\usepackage{graphicx}
\usepackage[utf8]{inputenc}
\usepackage{algorithm}
\usepackage{algorithmicx}
\usepackage{algpseudocode}
\usepackage{makecell}

\begin{document}
\begin{titlepage}
\title{\Huge \textbf{Structure de données II :\\
Rapport de l'étape préliminaire du projet}\\
\smallskip
\author{\emph{Groupe 5 :} \\
HUYLENBROECK Florent\\
DACHY Corentin\\}
}
\date{Année Académique 2018-2019\\
Bachelier en Sciences Informatiques\\
\vspace{1cm}
Faculté des Sciences, Université de Mons}
\maketitle
\end{titlepage}

\tableofcontents

\newpage
\section{Introduction}
Pour ce travail, voici les consignes qui nous on été demandées :\\
\begin{quote}
Pour se familiariser avec les arbres BSP (\emph{Binary space partitions}), il vous est demandé de réaliser l’exercice préliminaire suivant :\\[.5cm]
Etant donné un arbre BSP représentant une scène dans un plan (ensemble de segments) et deux points $x$ et $y$ dans ce plan, donnez un algorithme recursif en pseudo-code qui indique si le segment d’extrémités $x$ et $y$ appartient à la scène.  Veuillez accompagner votre algorithme :
\begin{itemize}
\item d’une explication de son fonctionnement; et
\item d’une discussion autour de sa complexité (ne vous limitez pas au pire des cas).
\end{itemize}
Nous convenons,  pour cet exercice,  que les segments contenus dans un même nœud de l’arbre BSP
sont stockés dans une liste chaînée.\\[.5cm]
\textbf{Remarque} : Cet exercice préliminaire n’est qu’une mise en route du projet.  Il ne sera pas nécessaire à la résolution du problème principal.
\end{quote}







\newpage
\section{Résolution}
\subsection{Pseudo-code}
\subsubsection{Algorithme belongsToScene}
\begin{algorithm}[H]
\caption{belongsToScene}
\begin{tabular}{lrl}
\multicolumn{3}{l}{\makecell[l]{Détermine si un segment dont les deux extrémités sont données sous forme de points en entrée appartient\\
exactement à l'arbre BSP donné en entrée.\\
Sert de phase d'initialisation à l'algorithme \emph{rechercher}.}}\\
&&\\
\textbf{Entrées} : &$BSP$ : &Partition de recherche binaire\\
& &On assume que chaque noeud contient l'équation de la droite qu'il décrit et\\
&&(facultatif) le segment qui lui est confondu, et chaque feuille contient un\\
&&segment décrit par une paire de points $S(S.x,S.y)$ et $S'(S'.x,S'.y)$.\\ 
&$A$ : &Point correspondant à une extrémité du segment.\\
& &Ce point a pour coordonnées ($A.x$,$A.y$).\\
&$B$ : &Point correspondant à l'autre extrémité du segment.\\
& &Ce point a pour coordonnées ($B.x$,$B.y$).\\[.25cm]
\textbf{Sorties} :& &Boléen, vrai si le segment appartient à la scène, faux sinon.\\[.25cm]
\textbf{Effets} :& &/\\[.25cm]
\end{tabular}
\begin{algorithmic}[1]
\Procedure{belongsToScene}{$BSP,A,B$}
\State $d\gets\Call{coefficientAngulaire}{A,B}$
\State \textbf{retourner} \Call{rechercher}{$BSP,A,B,d$}
\EndProcedure
\end{algorithmic}
\end{algorithm}

\subsubsection{Algorithme coefficientAngulaire}
\begin{algorithm}[H]
\caption{coefficientAngulaire}
\begin{tabular}{lrl}
\multicolumn{3}{l}{\makecell[l]{Calcule le coefficient angulaire d'un segment. Renvoie la valeur sentinelle $+\infty$ si le segment est vertical.}}\\
&&\\
\textbf{Entrées} : &$A$ : &Point, première extrémité du segment, de coordonnées ($A.x$,$A.y$).\\
&$B$ :&Point, deuxième extrémité du segment, de coordonnées ($B.x$,$B.y$).\\[.25cm]
\textbf{Sorties} :& &Le coeficient angulaire de la droite passant par $A$ et $B$.\\
& &Une valeur sentinelle $+\infty$ sera retournée si la pente est verticale.\\[.25cm]
\textbf{Effets} :& &/\\[.25cm]
\end{tabular}
\begin{algorithmic}[1]
\Procedure{coefficientAngulaire}{$A,B$}
\If{$A.x-B.x== 0$}
\State \textbf{retourner} $+\infty$
\Else
\State \textbf{retourner} $(A.y-B.y)/(A.x-B.x)$
\EndIf
\EndProcedure
\end{algorithmic}
\end{algorithm}

\subsubsection{Algorithme rechercher}
\begin{algorithm}[H]
\caption{rechercher}
\begin{tabular}{lrl}
\multicolumn{3}{l}{\makecell[l]{Recherche récursivement un segment dans un arbre BSP.\\
Pour cela, on localise d'abord tous les segments de l'arbre BSP dont une extrémité est la première\\
extrémité $P$ du segment recherché. On ne garde ensuite (dans une liste $S$) que les segments qui ont\\ 
le même coefficient angulaire $d$ que le segment recherché.\\[.25cm]
\begin{tabular}{rl}
\emph{Cas de base} :&Aucun segment de $S$ ne correspond à la recherche, $False$ est retourné.\\[.25cm]
\emph{Récursion} :&\makecell[lt]{pour chaque segment de $S$ si l'autre extrémité $P'$ de celui-ci correspond à la\\ 
deuxième extrémité $B$ du segment recherché, on s'arrête et on renvoie $True$.\\
Sinon, on regarde si le segment continue dans une autre feuille ou dans un autre\\
noeud, si oui, on rappelle l'algorithme avec $P'$ comme point de départ et $B$ comme\\
deuxième extrémité.}
\end{tabular}}}\\
&&\\
\textbf{Entrées} : &$BSP$ : &Partition de recherche binaire.\\
& $P$ : &Point, première extrémité du segment recherché dans le BSP\\
& $B$ : &Point, deuxième extrémité du segment recherché dans le BSP\\
& $d$ : &Entier (ou valeur sentinelle $+\infty$), coefficient angulaire du segment recherché.\\[.25cm]
\textbf{Sorties} :& &Boléen, vrai si le segment [PB] appartient à l'arbre BSP.\\[.25cm]
\textbf{Effets} :& &/\\[.25cm]
\end{tabular}
\begin{algorithmic}[1]
\Procedure{rechercher}{$BSP,P,B,d$}
\State $S[ ] \gets$ nouvelle liste vide
\State \Call{localiser}{$BSP,P,S[]$}
\State \Call{reduire}{$S[],d$}
\If{$S[]$ vide} 
\State \textbf{retourner} False
\Else 
\For{segment \textbf{in} $S[]$}
\State $P'\gets$ extrémité de $segment$ qui n'est pas $P$
\If{$P'== B$}
\State \textbf{retourner} True
\ElsIf{$P'$ sur un bord}
\State \textbf{retourner} \Call{rechercher}{$BSP,P',B,d$}
\EndIf
\EndFor
\EndIf
\State \textbf{retourner} False
\EndProcedure
\end{algorithmic}
\end{algorithm}

\subsubsection{Algorithme localiser}
\begin{algorithm}[H]
\caption{localiser}
\begin{tabular}{lrl}
\multicolumn{3}{l}{\makecell[l]{Recherche récursivement un point $P$ donné dans les segments d'un arbre BSP $root$.\\[.25cm]
\begin{tabular}{rl}
\emph{Cas de base} :&\makecell[lt]{$root$ est une feuille. Si le segment dans cette feuille contient $P$, alors $root$ est ajouté à\\
la liste de retour.\\[.25cm]}\\
\emph{Récursion} :&\makecell[lt]{On calcule $res$ le résultat de la résolution de l'équation de droite décrite par $root$ par\\
les coordonnées de $P$ pour trouver où $P$ se situe par rapport à celle-ci.\\
Si $P$ est à droite (\emph{resp.} gauche) de $root$ \emph{i.e.} $res>0$ (\emph{resp.} $res<0$), on\\
rappelle la fonction sur l'arbre de droite (\emph{resp.} gauche).\\
Si $P$ fait partie d'un segment contenu dans $root$ \emph{i.e.} $res=0$, alors $root$ est ajouté à\\
la liste de retour et on rappelle la fonction sur les deux fils de $root$.}\\
\end{tabular}}}\\
&&\\
\textbf{Entrées} : &$root$ : &Racine de la sous-partition de recherche binaire où l'on doit chercher.\\
&&On assume que $root$ possède un attribut $d$ étant l'équation de la droite décrite\\
&&par $root$ et $S$ qui contient le segment ($root$ est une feuille) ou les segments\\
&&($root$ est un noeud) contenus dans $root$, sous la forme de deux points\\
&& par segments (s'il y en a plusieurs, ils seront contenus dans une liste chainée).\\
&&$root+$ représente le sous-arbre au-dessus de $root.d$, et\\
&&$root-$ représente le sous-arbre en dessous de $root.d$ .\\
&$P$ :&Point que l'on recherche, de coordonnées ($P.x$,$P.y$).\\
&$return[]$ :&Liste des segments (paires de points) contenant le point recherché.\\[.25cm]
\textbf{Sorties} :& &/\\[.25cm]
\textbf{Effets} :& &Les segments contenant $P$ ont été ajoutés à $return[]$ .\\[.25cm]
\end{tabular}
\begin{algorithmic}[1]
\Procedure{localiser}{$root,P,return[]$}
\If{$root$ est une feuille}
\If{$P\in root.S$}
\State ajouter $root$ dans $return[]$
\EndIf
\Else
\State $res\gets$ résultat de la résolution de $root.d$ par le point $P.x$ et $P.y$
\If{$res>0$}
\State \Call{localiser}{$root+,P,return[]$}
\ElsIf{$res<0$}
\State \Call{localiser}{$root-,P,return[]$}
\Else
\For{$segment$ in $root.S$}
\If{$P\in segment$}
\State ajouter $segment$ dans $return[]$
\EndIf
\EndFor
\State\Call{localiser}{$root+,P,return[]$}
\State\Call{localiser}{$root-,P,return[]$}
\EndIf
\EndIf
\EndProcedure
\end{algorithmic}
\end{algorithm}

\subsubsection{Algorithme reduire}
\begin{algorithm}[H]
\caption{reduire}
\begin{tabular}{lrl}
\multicolumn{3}{l}{Réduit un ensemble de segments pour ne garder que ceux qui ont un coefficient angulaire donné.}\\
&&\\
\textbf{Entrées} : &$S[]$ : &Liste de segments à réduire.\\
& $d$ : &Entier (ou valeur sentinelle $+\infty$), coefficient angulaire du segment recherché.\\[.25cm]
\textbf{Sorties} :& &/\\[.25cm]
\textbf{Effets} :& &La liste $S[]$ ne contient plus que les segments qui ont un coefficient angulaire $d$.\\[.25cm]
\end{tabular}
\begin{algorithmic}[1]
\Procedure{reduire}{$S[],d$}
\ForAll{éléments $s$ de $S[]$}
\State $sd\gets$\Call{coefficientAngulaire}{$s.x,s.y$}
\If{$sd\neq d$}
\State retirer $s$ de $S[]$
\EndIf
\EndFor
\EndProcedure
\end{algorithmic}
\end{algorithm}

\newpage
\subsection{Discussion de la complexité}
Dans l'étude de la complexité de nos algorithmes, nous allons nommer :
\begin{itemize}
\item $s$ le nombre de segments dans la scène.
\item $h$ la hauteur de l'arbre BSP.
\item $l$ la largeur de l'arbre BSP.
\end{itemize}
\subsubsection{Algorithme coefficientAngulaire}
Dans le pire des cas comme dans un cas moyen, l'appel à cet algorithme se fait en $O(1)$.
\subsubsection{Algorithme reduire}
Dans le pire des cas, $S[]$ contient tous les segments de la scène. Ce cas n'est envisageable dans notre procédure que lorsque tous les segments de la scène ont une extrémité en commun. Il s'agira alors d'une complexité en $O(s)$ (on suppose que l'on retire les segments de la liste par index, donc en $O(1)$).\\[.5cm]
Dans le cas moyen par contre, $S[]$ ne contiendra que quelques segments. Le corps de la boucle ne contenant que des instructions en $O(1)$, l'algorithme aura pour complexité $O($nombre de segments dans $S[])$.\\[.5cm]
Pour la suite de la discussion, nous considérerons une complexité de $O(s)$ pour cet algorithme. Nous considérerons aussi que, dans le pire des cas, la liste $S[]$ contiendra, après application de la fonction, $s$ segments, contre $1$ dans un cas moyen.
\subsubsection{Algorithme localiser}
Chaque appel récursif de cet algorithme "descend" d'un noeud dans l'arbre. La condition d'arrêt est que l'on atteigne une ou plusieurs feuilles. Les instructions autres que les appels à la fonction étant en $O(1)$ (affectation, ajout d'un élément à une liste) et que l'on considère (lignes $13$:$17$) que $root.S$ contient dans le pire des cas tous les segments de la scène, et dans un cas moyen un seul segment. On aura, dans le pire des cas une complexité en $O(h+s)$, et dans un cas moyen $O(h)$.\\[.5cm]
\emph{Note : }il est possible qu'une récursion appelle deux fois la fonction (\emph{cf} lignes $18$:$19$, mais cela est négligeable car $O(2h)\in O(h)$.
\subsubsection{Algorithme rechercher}
Dans le pire des cas, le segment recherché traverse tout l'arbre BSP. On aura donc un nombre d'appels récursifs égal au nombre $l$ de feuilles de l'arbre. Notons que nous nous sommes permis d'écrire l'instruction "\emph{\textbf{else if }$P'$ sur un bord \textbf{then}}" (ligne $12$) à la place d'en écrire l'algorithme. Un tel algorithme aurait une complexité en $O(h)$. On a donc une complexité de $O(h+s+s$ (lignes $3$:$4$) $+ l\cdot s\cdot h$ (lignes $7$:$18$ si $S[]$ contient tous les segments de la scène)$)$. Donc $O(l\cdot s\cdot h)$.\\[.5cm]
Dans un cas moyen la complexité sera $O(l\cdot h)$ par les considérations précédentes.
\subsubsection{Algorithme belongsToScene}
Nous avons donc, au final, un algorithme ayant comme complexité dans le pire des cas $O(1+h+s+s\cdot l\cdot h)$ donc $O(s\cdot l\cdot h)$ et dans un cas moyen $O(1+h+s+l\cdot h)$ donc $O(l\cdot h)$.
\end{document}