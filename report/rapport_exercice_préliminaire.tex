\documentclass[10pt]{article}
\usepackage{fullpage}
\usepackage{graphicx}
\usepackage[utf8]{inputenc}
\usepackage{algorithm}
\usepackage{algorithmicx}
\usepackage{algpseudocode}

\begin{document}
\begin{titlepage}
\title{\Huge \textbf{Structure de données II :\\
Rapport de l'étape préliminaire du projet}\\
\smallskip
\author{\emph{Groupe 5 :} \\
HUYLENBROECK Florent\\
DACHY Corentin\\}
}
\date{Année Académique 2018-2019\\
Bachelier en Sciences Informatiques\\
\vspace{1cm}
Faculté des Sciences, Université de Mons}
\maketitle
\end{titlepage}

\tableofcontents

\newpage
\section{Introduction}
Pour ce travail, voici les consignes qui nous on été demandées :\\
\begin{quote}
Pour se familiariser avec les arbres BSP (\emph{Binary space partitions}), il vous est demandé de réaliser l’exercice préliminaire suivant :\\[.5cm]
Etant donné un arbre BSP représentant une scène dans un plan (ensemble de segments) et deux points $x$ et $y$ dans ce plan, donnez un algorithme recursif en pseudo-code qui indique si le segment d’extrémités $x$ et $y$ appartient à la scène.  Veuillez accompagner votre algorithme :
\begin{itemize}
\item d’une explication de son fonctionnement; et
\item d’une discussion autour de sa complexité (ne vous limitez pas au pire des cas).
\end{itemize}
Nous convenons,  pour cet exercice,  que les segments contenus dans un même nœud de l’arbre BSP
sont stockés dans une liste chaînée.\\[.5cm]
\textbf{Remarque} : Cet exercice préliminaire n’est qu’une mise en route du projet.  Il ne sera pas nécessaire à la résolution du problème principal.
\end{quote}







\newpage
\section{Résolution}
\subsection{Pseudo-code}
\begin{algorithm}[!h]
\caption{belongsToScene}
\begin{tabular}{lrl}
\multicolumn{3}{l}{Détermine si un segment dont les deux extrêmités données sous formes de points en entrée appartient}\\
\multicolumn{3}{l}{à l'arbre BSP donné en entrée.}\\
&&\\
\textbf{Entrées} : &$BSP$ : &Partition de recherche binaire\\
& &On assume que chaque noeud contient l'équation de la droite qu'il décrit et (facultatif) le\\ 
& &segment qui lui est confondu, et chaque feuille contient un segment, décrit par\\
& &une paire de points $S(S.x,S.y)$ et $S'(S'.x,S'.y)$.\\ 
&$A$ : &Point correspondant à une extremité du segment.\\
& &Ce point a pour coordonnées ($A.x$,$A.y$).\\
&$B$ : &Point correspondant à l'autre extremité du segment.\\
& &Ce point a pour coordonnées ($B.x$,$B.y$).\\
\textbf{Sorties} :& &Boléen, vrai si le segment appartient à la scene, faux sinon.\\
\textbf{Effets} :& &/
\end{tabular}
\begin{algorithmic}[1]
\Procedure{belongsToScene}{$BSP,A,B$}
\State $d\gets\Call{coefficientAngulaire}{A,B}$
\State \textbf{retourner} \Call{rechercher}{$BSP,A,B,d$}
\EndProcedure
\end{algorithmic}
\end{algorithm}

\begin{algorithm}
\caption{coefficientAngulaire}
\begin{tabular}{lrl}
\multicolumn{3}{l}{Calcule le coefficient angulaire d'un segment.}\\
&&\\
\textbf{Entrées} : &$A$ : &Racine de la sous-partition de recherche binaire que l'on doit chercher\\
&$B$ :&Point que l'on recherche.\\
\textbf{Sorties} :& &Le coeficient angulaire de la droite passant par $A$ et $B$.\\
& &Une valeur sentinelle $+\infty$ sera retournée si la pente est verticale.\\
\textbf{Effets} :& &/
\end{tabular}
\begin{algorithmic}[1]
\Procedure{coeficientAngulaire}{$A,B$}
\If{$A.x-B.x== 0$}
\State \textbf{retourner} $+\infty$
\Else
\State \textbf{retourner} $(A.y-B.y)/(A.x-B.x)$
\EndIf
\EndProcedure
\end{algorithmic}
\end{algorithm}

\newpage

\begin{algorithm}[!h]
\caption{rechercher}
\begin{tabular}{lrl}
\multicolumn{3}{l}{Recherche récursivement un segment dans un arbre BSP.}\\
&&\\
\textbf{Entrées} : &$BSP$ : &Partition de recherche binaire\\
& &On assume que chaque noeud contient l'équation de la droite qu'il décrit et (facultatif) le\\ 
& &segment qui lui est confondu, et chaque feuille contient un segment, décrit par\\
& &une paire de points $S(S.x,S.y)$ et $S'(S'.x,S'.y)$.\\ 
& $P$ : &Point, premiére extremité du segment recherché dans le BSP\\
& $B$ : &Point, deuxième extremité du segment recherché dans le BSP\\
& $d$ : &Entier (ou valeur sentinelle $+\infty$), coeficient angulaire du segment recherché.\\
\textbf{Sorties} :& &Boléen, vrai si le segment PB appartient au BSP\\
\textbf{Effets} :& &/
\end{tabular}
\begin{algorithmic}[1]
\Procedure{rechercher}{$BSP,P,B,d$}
\State $S[ ] \gets$ nouvelle liste vide
\State \Call{localiser}{$BSP,P,S[]$}
\State \Call{reduire}{$S[],d$}
\If{$S[]$ vide} 
\State \textbf{retourner} False
\Else 
\For{segment \textbf{in} $S[]$}
\State $P'\gets$ extremité de $segment$ qui n'est pas $P$
\If{$P'== B$}
\State \textbf{retourner} True
\ElsIf{$P'$ sur un bord}
\State \textbf{retourner} \Call{rechercher}{$BSP,P',B,d$}
\EndIf
\EndFor
\EndIf
\State \textbf{retourner} False
\EndProcedure
\end{algorithmic}
\end{algorithm}
\newpage
\begin{algorithm}[!h]
\caption{localiser}
\begin{tabular}{lrl}
\multicolumn{3}{l}{Recherche récursivement un point donné dans les segments d'un arbre BSP}\\
&&\\
\textbf{Entrées} : &$root$ : &Racine de la sous-partition de recherche binaire où l'on doit chercher\\
&$P$ :&Point que l'on recherche.\\
&$return[]$ :&Liste contenant les résultats de la recherche.\\
\textbf{Sorties} :& &/\\
\textbf{Effets} :& &La liste $return[]$ est mise à jour.
\end{tabular}
\begin{algorithmic}[1]
\Procedure{localiser}{$root,P,return[]$}
\If{$root$ est une feuille}
\If{$P\in root$}
\State ajouter $root$ dans $return[]$
\EndIf
\Else
\State $res\gets$ résultat de la résolution de l'équation de la droite décrite par $root$ avec $P.x$ et $P.y$ 
\If{$res\geq 0$}
\State \Call{localiser}{$root+,P,return[]$}
\ElsIf{$res\leq 0$}
\State \Call{localiser}{$root-,P,return[]$}
\Else
\If{$P\in root$}
\State ajouter $root$ dans $return[]$
\EndIf
\State\Call{localiser}{$root+,P,return[]$}
\State\Call{localiser}{$root-,P,return[]$}
\EndIf
\EndIf
\EndProcedure
\end{algorithmic}
\end{algorithm}
\begin{algorithm}
\caption{reduire}
\begin{tabular}{lrl}
\multicolumn{3}{l}{Réduit un ensemble de segments pour ne garder que ceux qui ont un coefficient angulaire donné.}\\
&&\\
\textbf{Entrées} : &$S[]$ : &Ensemble de segments à réduire.\\
& $d$ : &Entier (ou valeur sentinelle $+\infty$), coeficient angulaire du segment recherché.\\
\textbf{Sorties} :& &/\\
\textbf{Effets} :& &La liste $S[]$ ne contient plus que les segments qui ont un coefficient angulaire $d$.
\end{tabular}
\begin{algorithmic}[1]
\Procedure{reduire}{$S[],d$}
\ForAll{elements $s$ de $S[]$}
\State $sd\gets$\Call{coeficientAngulaire}{$s.A,s.B$}
\If{$sd\neq d$}
\State retirer $s$ de $S[]$
\EndIf
\EndFor
\EndProcedure
\end{algorithmic}
\end{algorithm}
\newpage

\subsection{Discussion de la complexité}

\end{document}