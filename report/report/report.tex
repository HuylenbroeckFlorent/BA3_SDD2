\documentclass[11pts]{article}

\usepackage[utf8]{inputenc}
\usepackage[T1]{fontenc}
\usepackage{fullpage}
\usepackage{graphicx}
\usepackage{longtable}
\usepackage{xcolor}
\usepackage{listings}
\usepackage{calrsfs}

\title{Structure de données II\\Rapport de projet}
\author{HUYLENBROECK Florent\\DACHY Corentin\\BA3 Info}
\date{15 avril 2019}

\begin{document}
\maketitle
\newpage
\tableofcontents
\newpage
\section{Introduction}
\subsection{Notions théoriques}
\begin{itemize}
\item Une \textbf{scène} est un ensemble de segments colorés dans un repère fini. Dans ce rapport, nous nommerons une scène $S$.
\item Un \textbf{BSP} (\emph{Binary Space Partition} ou \emph{partition binaire de l'espace}) est un arbre binaire permettant de représenter une scène en fragmentant celle-ci en sous-espaces convexes, lesquels contiennent au plus un segment.
Son fonctionnement est le suivant :\\[.2cm]
Soit $v$ la racine de l'arbre $\tau$.
\begin{itemize}
\item Si $v$ est une feuille, alors $S(v)$ (sous-ensemble de segments de la scéne contenus dans $v$) contient $0$ ou $1$ segment.
\item Sinon $v$ est un noeud interne du BSP. Dans ce cas, $v$ contient une équation de droite de la forme $\ell\equiv ax+by+c=0$ et un ensemble $S(v)$\\
Le fils gauche $v_{gauche}$ de $v$ sera la racine du sous-BSP $\tau^-$ décrivant la partie de la scène située sous la droite $\ell$, et le fils droit $v_{droit}$ de $v$ sera la racine du sous-BSP $\tau^+$ décrivant la partie de la scéne située au-dessus de la droite $\ell$.\\
$S(v)$ contient les segments totalement inclus à la droite $\ell$. Leur nombre n'est pas limité comme dans les feuilles.
\end{itemize}
Si une droite de séparation $\ell_v$ d'un noeud interne $v$ intersecte un segment en un point autre qu'une extrémité, ce segment est coupé en deux parties et celles-ci sont ajoutées dans $\tau^-$ et $\tau^+$ en conséquence.
\item La \textbf{taille} d'un BSP est le nombre total de segments contenus dans chacuns des noeuds de celui-ci.
\item Une \textbf{heuristique} de construction d'un BSP intervient au moment de choisir le segment de la scène par lequel passera la prochaine droite de séparation $\ell$.
\end{itemize}

\subsection{Enoncé du problème}
Pour ce projet, il nous a été demandé à partir de fichier \emph{.txt} au format Scene2D de :
\begin{itemize}
\item Construire un BSP de la scène décrite dans le fichier.
\item Appliquer l'algorithme du peintre sur ce BSP.
\item Fournir une application ayant deux modes, console et graphique, afin de tester différentes heuristiques de construction pour les BSP. L'application console doit aussi mesurer la hauteur/taille du BSP, le temps CPU pris par la construction et par l'excecution de l'algorithme du peintre.
\item Trois heuristiques nous ont été imposée pour ce projet :
\begin{itemize}
\item RANDOM - Le segment est choisi au hasard parmis $S(v)$.
\item ORDERED - Le segment choisi est le premier dans $S(v)$.
\item FREE\_SPLIT - Le segment choisi est le premier dans $S(v)$ dont les deux extrêmités sont chacune sur une droite de séparation $\ell$. Si aucun segment satisfaisant ces condtitions n'est trouvé dans $S(v)$, alors le segment est choisi au hasard dans $S(v)$.
\end{itemize}
\end{itemize}

\newpage
\section{Calcul de la complexité de l'algorithme du peintre}


\newpage
\section{Comparaison des heuristiques}


\newpage
\section{Mode d'emploi du logiciel de test}
\subsection*{Mode graphique}
Ouvrir un terminal, se déplacer dans le dossier \emph{code} du logiciel et executer la commande :
\begin{lstlisting}[language=bash]
$ ant testgui
\end{lstlisting}
Le logiciel consiste en une fenêtre séparée en deux parties. La partie du haut permet de visualiser le BSP et celle du bas ce que voit l'oeil.
Afin d'ouvrir un BSP, il faut se rendre dans le menu \emph{Actions} $\rightarrow$ \emph{Open} et choisir un fichier Scene2D valide.
Il est possible de changer l'heuristique de construction via le menu \emph{Heuristics}.
\subsection*{Mode console}
Ouvrir un terminal, se déplacer dans le dossier \emph{code} du logiciel et executer la commande :
\begin{lstlisting}[language=bash]
$ ant test
\end{lstlisting}
Il est possible de spécifier les conditions de départ des test grâce aux paramètres suivants :
\begin{itemize}%[\itemsep=0em]
\item \emph{path} - Le chemin jusqu'au fichier Scene2D. La valeur par défaut est "octogone.txt".
\item \emph{n} - Le nombre de répétitons du test. La valeur par défaut est 100.
\item \emph{ex} - L'abscisse de l'oeil. La valeur par défaut est 0.
\item \emph{ey} - L'ordonnée de l'oeil. La valeur par défaut est 0.
\end{itemize}
Pour modifier ces paramètres, il faut utiliser le modificateur \emph{-D}.
\begin{lstlisting}[language=bash]
$ ant test -Dpath=somepath.txt
\end{lstlisting}
Pour modifier plusieurs paramétres, plusieurs \emph{-D} sont nécessaires.
\begin{lstlisting}[language=bash]
$ ant test -Dpath=somepath.txt -Dn=1000 -Dex=10.05 -Dey=3.141
\end{lstlisting}
\subsection*{Javadoc}
Ouvrir un terminal, se déplacer dans le dossier \emph{code} du logiciel et executer la commande :
\begin{lstlisting}[language=bash]
$ ant doc
\end{lstlisting}


\newpage
\section{Conclusion}

\end{document}